\documentclass[es]{uc3mreport}



\usepackage{import}  % import TeX files
\usepackage{graphicx}



% general config

\graphicspath{{img/}}  % Images folder
\addbibresource{references.bib}  % bibliography file

\degree{Grado en Ingeniería Informática}
\subject{Visión Artificial}
% \shortsubject{LTX}  % optional
\academicyear{2024-2025}
\group{85}
\author{
    César López Mantecón  -- 100472092\\
    Álvaro Guerrero Espinosa -- 100472294\\
    José Antonio Verde Jiménez -- 100069420}
% \team{Equipo 69}  % optional

\shortauthor{\abbreviateauthor{César}{L.M.}, \abbreviateauthor{Álvaro}{G.E.} y \abbreviateauthor{José A.}{V.J.}}
\lab{Práctica 2}
\title{Inversión de Neural Style}
% \shorttitle{La mejor memoria de la historia}  % optional
\professor{Fernando Díaz de María\\Leire Paz Arbaizar}  % optional


% report

\begin{document}
    \makecover

%    \tableofcontents
    % \listoffigures
    % \listoftables

    % contenidos
    \begin{report}
        \section{Propuesta}
        El objetivo de este proyecto es el estudio de la capacidad de la red
        para deshacer la transformación de estilo aplicando la imagen de
            contenido original como imagen de estilo a una previamente
            transformada. Es decir, se ha transformado una imagen aplicando un
            estilo y se ha tratado de recuperar la imagen original aplicando
            esta última como estilo.

        La imagen empleada para la primera transformación será la que se muestra en la figura~\ref{original}.


        \section{Metodología}
        Para estudiar distintintas alternativas, cada uno de los miembros del
        equipo ha trabajando por separado y con un estilo diferente. Con esto, se ha buscado de forma
        independiente una solución al problema planteado. A continuación se
        expone qué alternativas ha explorado cada integrante.

            

            \subsection{César López Mantecón}
            Se ha explorado la propuesta variando el número de capas
            convolucionales y limitando la imagen a 128 píxeles. Por último, se
            ha tratado de recuperar una imagen lo más parecida a la original
            empleando otra parecida como imagen de estilo.

            La imagen de estilo utilizada para generar la primera
            transformación (la que se tratará de deshacer) es una pintura del
            artista catalán Joan Miró (figura~\ref{joan-miro}).

            La imagen intermedia, resultado de aplicar la
            imagen~\ref{joan-miro} como estilo a la orignal se peude ver
            en~\ref{intermedio-clm}.

            \subsection{Álvaro Guerrero Espinosa}

            \subsection{José Antonio Verde Jiménez}


        \section{Resultados}

            En el experimento desarrollado por César se ha observa que un mayor
            número de capas convolucionales es capaz de recuperar más detalles
            de la imagen original. Además, la limitación a 128 píxeles muestra
            una mayor introducción de ruido en las transformaciones que perdura
            hasta el resultado final, como se puede ver en~\ref{final-clm-1}.

            Además, al emplear una imagen de otra ciudad (figura~\ref{city-2})
            se ha observado que el modelo es capaz de recuperar mucho detelle
            de la imagen original. No obstante, los colores varían,
            predominando tonos más cálidos frente a las tonalidades frías de la
            imagen original~\ref{final-clm-2}.


    \section{Figuras}

    \begin{figure}
        \includeinkscape[width=0.45\textwidth]{tar3.png}
        \caption{Imagen Original}
        \label{original}
    \end{figure}

    \begin{figure}
        \includeinkscape[width=0.45\textwidth]{joan_miro.png}
        \caption{Imagen de estilo para el experimento de César}
        \label{joan-miro}
    \end{figure}

    \begin{figure}
        \includeinkscape[width=0.45\textwidth]{intermedio.png}
        \caption{Imagen intermedia en el experimento de César}
        \label{intermedio-clm}
    \end{figure}

    \begin{figure}
        \includeinkscape[width=0.45\textwidth]{final-clm1.png}
        \caption{Imagen final 1 - César}
        \label{final-clm-1}
    \end{figure}

    \begin{figure}
        \includeinkscape[width=0.45\textwidth]{other_city.png}
        \caption{Segunda imagen de estilo - César}
        \label{city-2}
    \end{figure}

    \begin{figure}
        \includeinkscape[width=0.45\textwidth]{final-clm2.png}
        \caption{Imagen final 1 - César}
        \label{final-clm-2}
    \end{figure}

    \end{report}


%    % bibliography
%    \label{bibliography}
%    \part{Referencias}
%    \printbibliography
    % \printbibliography[heading=bibintoc,title={Referencias}]  % alternative

    % appendices
%    \begin{appendices}
%        \part{Apéndices}  % optional
%        \section{Mi apéndice}
%        \lipsum[1]
%    \end{appendices}

\end{document}
